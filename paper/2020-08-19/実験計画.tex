\documentclass[a4j]{jarticle}
\begin{document}
\begin{center}
実験計画
\end{center}
\section{背景}
10 回の ping 応答遅延計測を 10 ミリ秒間隔で行い,これを 15 秒毎に繰り返す実験では,10 ミリ秒間隔で行う 10 回の計測の一回目が大きく二つ目以降から小さくなっていく傾向が見て取れた.
これは,二回目以降の計測では一回目の計測の影響で eNodeB でのリソース割り当てが高速化しているからではないかと考えられる.
そこでここでは,この高速化を引き起こす原因が ICMP のプロトコルに依存するものなのかどうかを調べるため,FTP による通信をバックグラウンドで行いながらの以下の二つの実験を行う.
\section{予備実験}
FTP によるファイル転送が上りでどれぐらいの速さであるかを調べた.
1M バイトのファイルを AWS サーバに送信するのに要した時間は,ネットワークが混雑していると考えられる 12 時台では約 4 秒,空いていると考えられる 15 時台では 0.2 秒であった.
以降の実験は,1M バイトを転送するのに要する時間は 0.2 秒として設計している.
しかし,念のため実験実施直前にもこの所要時間の検証を行い,通信量を必要最小限になるようにする.
\section{実験1}
FTP 通信下においても 10 ミリ秒間隔の計測が減少傾向を示すのかを調べる.
もし一回目の計測から小さな値を取るのであれば,ICMP に限らず他に通信が行われている状況での ping 応答遅延は小さくなると考えられる.
\subsection{実験手順}
\begin{enumerate}
\item 計測開始時に一台の Raspberry Pi から AWS サーバに FTP 接続を行う
\item 10 M バイト(約 2 秒間)の送信を開始する
\item 2. の開始の 0.5 秒後から 10 ミリ行間隔で 10 回の ping 計測を行う.同時に各 ping コマンドの実行直前の時刻も取得する
\item 15 秒後に 2. に戻る.終了条件は 1. から 20 分経過した場合とする
\end{enumerate}
この実験での総通信量は約 800M バイトと見込まれ,総計測点は 800 である.
\section{実験2}
FTP 通信下においても 15 秒間隔での ping 計測値が 70 ミリ秒から 90 ミリ秒までの間に多く分布するのかどうかを調べる.
もし 10 ミリ秒間隔での計測値のように 30 ミリ秒から 50 ミリ秒の間に多く分布するのであれば,他の通信が行われている状況での応答遅延の分布帯は小さなものになると考えられる.
\subsection{実験手順}
\begin{enumerate}
\item 計測開始時に一台の Raspberry Pi から AWS サーバに FTP 接続を行う
\item 7 M バイト(約 1.4 秒)の送信を開始する
\item 2. の開始の 0.5 秒後に ping 計測を行う.同時に各 ping コマンドの実行直前の時刻も取得する
\item 15 秒後に 2. に戻る.終了条件は 1. から 1 時間経過した場合とする
\end{enumerate}
この実験での総通信量は約 1.68 G バイトと見込まれ,総計測点は 240 である.
\end{document}
