\documentclass[a4j]{jarticle}
\usepackage[dvipdfmx]{graphicx}
\usepackage{amssymb}
\usepackage{subfigure}
\usepackage{longtable}

\begin{document}
\begin{table}[t]
\begin{center}
{\large 産業用モニタリングシステムにおける障害検知手法の提案}\\
令和 2 年 8 月 4 日\\
山本 航平
\end{center}
\end{table}

進捗報告
\begin{itemize}
\item FTP を用いない一日を通じた 15 秒間隔での計測実験を終えました.
\item FTP を用いた計測実験の準備を進めています.
\item 一日を通じた計測実験で見られる,低遅延帯が時間経過に伴い増加傾向を示す区間の時間的長さを調べました.
\item 10 ミリ秒間隔で行った計測における変動値の分布を調べました.
\end{itemize}

\section{FTP を用いない計測データ}
6 月 23 日(木)から 7 月 22 日(月)までの約一か月間,一日を通じ 15 秒間隔で ping による応答遅延を計測した.
これにより得られた各計測時刻における ping 応答遅延の値を一日ごとにわけて図 \ref{all} に示す.
\begin{figure}[tb]
\begin{center}
\subfigure[6 月 23 日(木)]{
\includegraphics[width=0.33\hsize]{C:/master/mstudy/analysis/long/plot-6_23.pdf}
}~
\subfigure[6 月 24 日(金)]{
\includegraphics[width=0.33\hsize]{C:/master/mstudy/analysis/long/plot-6_24.pdf}
}~
\subfigure[6 月 25 日(土)]{
\includegraphics[width=0.33\hsize]{C:/master/mstudy/analysis/long/plot-6_25.pdf}
}\\
\subfigure[6 月 26 日(日)]{
\includegraphics[width=0.33\hsize]{C:/master/mstudy/analysis/long/plot-6_26.pdf}
}~
\subfigure[6 月 27 日(月)]{
\includegraphics[width=0.33\hsize]{C:/master/mstudy/analysis/long/plot-6_27.pdf}
}~
\subfigure[6 月 28 日(火)]{
\includegraphics[width=0.33\hsize]{C:/master/mstudy/analysis/long/plot-6_28.pdf}
}\\
\subfigure[6 月 29 日(水)]{
\includegraphics[width=0.33\hsize]{C:/master/mstudy/analysis/long/plot-6_29.pdf}
}~
\subfigure[6 月 30 日(木)]{
\includegraphics[width=0.33\hsize]{C:/master/mstudy/analysis/long/plot-6_30.pdf}
}~
\subfigure[7 月 1 日(金)]{
\includegraphics[width=0.33\hsize]{C:/master/mstudy/analysis/long/plot-7_1.pdf}
}\\
\subfigure[7 月 2 日(土)]{
\includegraphics[width=0.33\hsize]{C:/master/mstudy/analysis/long/plot-7_2.pdf}
}~
\subfigure[7 月 3 日(日)]{
\includegraphics[width=0.33\hsize]{C:/master/mstudy/analysis/long/plot-7_3.pdf}
}~
\subfigure[7 月 4 日(月)]{
\includegraphics[width=0.33\hsize]{C:/master/mstudy/analysis/long/plot-7_4.pdf}
}
\caption{計測結果}
\end{center}
\end{figure}
\end{document}